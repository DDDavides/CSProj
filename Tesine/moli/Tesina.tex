-Fare riferimenti al tutorial
    -resource oriented programming
        -singola risorsa
        -collezione di nft
    -capabilities
-Fare riferimenti agli standard
    -nft con le sue funzioni:
        -id
        -withdraw
        -deposit
        -collectionpublic
-Fare riferimenti alla documentazione ufficiale:
    -flow.json per le configurazioni
    -deploy project
    -send signed transactions
-Fare riferimenti a crypto kitties

concetti:
    -prepare di una transazione è dove ha full access all'account che l'ha firmata e dove 
    memorizzati e rimossi e usate per creare capabilities le risorse

-Vendita di NFT con FT:
    -evitare che quando si mette in vendita si perda l'NFT, finchè non viene venduto si ha il possesso ancora
    -ogni utente elenca una lista di vendite dal proprio account anzichè avere un approccio centralizzato
    -ogni utente possiede risorsa SaleCollection che ha:
        -collezione di NFT dell'owner da cui andare a prendere gli NFT da vendere
        -dizionario di prezzi degli NFT da vendere
        -vault dell'owner dove andare a depositare i soldi una volta venduto l'NFT
        con le seguenti funzioni:
            -cancelSale per eliminare la vendita di un NFT
            -listForSale che elenca tutti gli NFT in vendita in quella collezione
            -changePrice per cambiare il prezzo a un NFT che è in vendita
            -purchase che permette di acquistare da parte di un esterno recipient con dei 
            buytokens un NFT in vendita in quella collezione
            -idPrice per avere il prezzo di un singolo NFT da parametor
            -getIDs per ottenere tutti i prezzi degli NFT in vendita

    -transazione create_sale:
        -ottiene i riferimenti alla collezione da voler linkare per la vendita e al
        vault in cui poter ricevere i soldi
        -crea o ottiene un riferimento a una SaleCollection e la linka con una capability
        -invoca il meto listForSale per il token passato come parametro al prezzo desiderato

    -transazione purchase_nft:
        -nel prepare prende dall'account che firma la capability della collezione
            poi si prende il riferimento al vault dell'account che deve pagare e poi prende dall'account
            owner del token la SaleCollection e controlla che sia presente il token andandosi a prendere 
            il price e infine ritira i soldi dal vaultref
        -nell'execute invoca il metodo purchase per quel token inviandolo a quel ricevente con la capability 
            alla collection da cui prendere il token