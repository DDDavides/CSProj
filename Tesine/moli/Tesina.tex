-Fare riferimenti al tutorial
    -resource oriented programming
        -singola risorsa
        -collezione di nft
    -capabilities
-Fare riferimenti agli standard
    -nft con le sue funzioni:
        -id
        -withdraw
        -deposit
        -collectionpublic
-Fare riferimenti alla documentazione ufficiale:
    -flow.json per le configurazioni
    -deploy project
    -send signed transactions
-Fare riferimenti a crypto kitties

-Vendita di NFT con FT:
    -evitare che quando si mette in vendita si perda l'NFT, finchè non viene venduto si ha il possesso ancora
    -ogni utente elenca una lista di vendite dal proprio account anzichè avere un approccio centralizzato
    -ogni utente possiede risorsa SaleCollection che ha:
        -collezione di NFT dell'owner da cui andare a prendere gli NFT da vendere
        -dizionario di prezzi degli NFT da vendere
        -vault dell'owner dove andare a depositare i soldi una volta venduto l'NFT
        con le seguenti funzioni:
            -cancelSale per eliminare la vendita di un NFT
            -listForSale che elenca tutti gli NFT in vendita in quella collezione
            -changePrice per cambiare il prezzo a un NFT che è in vendita
            -purchase che permette di acquistare da parte di un esterno recipient con dei 
            buytokens un NFT in vendita in quella collezione
            -idPrice per avere il prezzo di un singolo NFT da parametor
            -getIDs per ottenere tutti i prezzi degli NFT in vendita