Nel corso dello svolgimento del progetto sono stati riscontrati alcuni problemi relativi all'utilizzo di alcune tecnologie e ambienti di sviluppo messi a disposizione dalla blockchain che ci hanno spinto a trovare soluzioni differenti. 

Il primo problema riscontrato ha generato una catena di problemi ed è relativo all'ambiente di sviluppo messo a disposizione da flow (\textit{playground}) che, per quanto semplice ed intuitivo, non presenta la possibilità di importare file e di lavorare in più persone su di uno stesso progetto. Per questo motivo abbiamo deciso di spostarci su Visual Studio Code utilizzando la relativa estensione e la CLI (richiesta dall'estensione), la quale dovrebbe presentare un'interfaccia di sviluppo simile a quella di playground, ma in questo caso abbiamo riscontrato poca compatibilità tra l'estensione e l'emulatore il che ci ha spinto ad utilizzare l'estensione esclusivamente per l'autocompletamento ed il syntax highlighting mentre la CLI per interagire direttamente con l'emulatore.

Infine anche l'utilizzo della CLI ha generato qualche grattacapo dal momento che non risulta essere possibile inviare una transazione in maniera semplice con il comando \codeinline{send} nel momento in cui questa necessita di due o più firmatari.