La blockchain utilizzata per questo progetto è Flow, una valida alternativa a quelle già esistenti e che si distacca da le più note (e.g. Bitcoin) per una serie di innovazioni e cambiamenti apportati dai suoi creatori, la società Dapper Labs (creatrice anche di CryptoKitties). 

Lo sviluppo di Flow inizia nel 2019 quando le tecnologie concorrenti erano in crescita ed aumento, tra queste anche Ethereum cui riferimento è inevitabile poiché Flow prova a superarne i limiti e punti deboli espressi col tempo. L'obiettivo perseguito da Flow fa ausilio di un'architettura che si impegna a supportare una nuova generazione di giochi, apps e assets digitali, e affinché sia possibile, Flow apporta dei cambiamenti che inoltre rappresentano i suoi punti di forza: 

\begin{itemize}
    \item \textbf{Architettura multi ruolo}. Flow permette la coesistenza di nodi a cui è possibile assegnare ruoli differenti, ciascuno relativo alla realizzazione di un task. Esistono i seguenti 5 tipi di nodi: nodi di raccolta, consenso, esecuzione, verifica o accesso. La suddivisione di queste attività ha guidato la realizzazione di una blockchain veloce poiché sfrutta la parallelizzazione dei tasks, altrimenti sostituita da un approccio lineare e quindi più lento.
    \item \textbf{Programmazione orientata alle risorse}. Si introduce un linguaggio proprietario per la realizzazione di Smart Contracts (e non solo), stiamo parlando di \textit{Cadence}, un linguaggio semplice, sicuro e orientato alle risorse.
    \item \textbf{Ergonomia dello sviluppatore}. Sono messi a disposizione dei tools per aiutare lo sviluppo e i programmatori, a cui inoltre si rende più semplice, a differenza di Ethereum, la modifica e/o aggiornamento degli Smart Contracts.
    \item \textbf{HotStuff, un protocollo di consenso rapido}. L'algoritmo di consenso utilizzato da Flow nasce dall'esigenza di creare un protocollo BFT veloce, sicuro e scalabile, che possa puntare a produrre un blocco in poco tempo, così che la convalida delle transazioni sia quasi istantanea.
\end{itemize}

% Utente non fidarti del creatore dello smart contract ma dello smart contract code
% Puoi diventare un nodo anche con poco hw (hw non costoso)

\sezione{Cadence}
{capitoli/flow/cadence}

\sezione{Flow CLI}
{capitoli/flow/flow_cli}