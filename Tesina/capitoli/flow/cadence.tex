Sito loro

https://flow.com/post/resources-programming-ownership

cadence ha alcune best practices e design patterns e ci metti un rimando alla bibliografia
\subsection{Resources}
Stralcio di codice con concetti principali presi dal tutorial, come move operator, il fatto che sono uniche e non possono essere copiate. A differenza delle struct 

Collezione di risorse. in un path solo una risorsa che è la collezione e non posso avere più risorse nello stesso path oooo capito?!

https://developers.flow.com/cadence/tutorial/03-resources

\subsection{Accounts}
In flow ogni account può essere acceduto per mezzo di due differenti tipi che sono:
\begin{itemize}
	\item \codeinline{PublicAccount}: rappresenta la parte pubblica di un account, la quale può essere acceduta da qualsiasi codice tramite la funzione built-in di flow \codeinline{getAccount()} per mezzo dell'address dell'account cui si vuole accedere.
	\item \codeinline{AuthAccount}: rappresenta la parte autorizzata di un account, la quale ha pieno controllo sullo \textbf{storage}, sulla chiave pubblica e sul codice di tale account. Per questo transazione può accedere all'\codeinline{AuthAccount} solo di chi la firma come authorizer. Tale interazione viene definita nel codice della transazione come un parametro nella fase di \codeinline{prepare}.
\end{itemize}
Tutti gli account presentano quello che è uno \textbf{Storage}, nel quale possono essere immagazzinate sia le strutture dati (struct) che le risorse. Un oggetto nello storage viene immagazzinato sotto un \codeinline{Path}, il quale consiste in un dominio, un identificatore ed inizia con il carattere \codeinline{/}. Esistono tre tipi di Path che si riferiscono a tre diversi dominii dello storage che sono:
\begin{itemize}
	\item \codeinline{/Storage/}: con il relativo tipo \codeinline{StoragePath};
	\item \codeinline{/Public/}: con il relativo tipo \codeinline{PublicPath};
	\item \codeinline{/Private/}: con il relativo tipo \codeinline{PrivatePath}.
\end{itemize}
Nello StoragePath possono essere immagazzinati soltando gli oggetti, mentre sotto PublicPath e PrivatePath possono essere immagazzinate solo le Capabilities.

\subsection{Capabilities}
https://developers.flow.com/cadence/language/capability-based-access-control

https://developers.flow.com/cadence/tutorial/04-capabilities