% Classe appositamente creata per tesi di Ingegneria Informatica all'università Roma Tre
\documentclass{TesiDiaUniroma3}
% --- INIZIO dati relativi al template TesiDiaUniroma3
% dati obbligatori, necessari al frontespizio
\titolo{Flow NFT}
\autoreprimo{Davide Galletti}
\matricolaprima{533152}
\autoresecondo{Davide Gattini}
\matricolaseconda{540202}
\autoreterzo{Davide Molitierno}
\matricolaterza{537969}
\github{https://github.com/DDDavides/CSProj}
% --- FINE dati relativi al template TesiDiaUniroma3

% --- INIZIO richiamo di pacchetti di utilità. Questi sono un esempio e non sono strettamente necessari al modello per la tesi.
\usepackage[plainpages=false]{hyperref}	% generazione di collegamenti ipertestuali su indice e riferimenti
\usepackage[all]{hypcap} % per far si che i link ipertestuali alle immagini puntino all'inizio delle stesse e non alla didascalia sottostante
\usepackage{amsthm}	% per definizioni e teoremi
\usepackage{amsmath}	% per ``cases'' environment
% --- FINE richiamo di pacchetti di utilità

\usepackage{listings} % per scrittura di codice

\usepackage{xcolor}
\definecolor{bluegray}{rgb}{0.4, 0.6, 0.8}
\definecolor{coolblack}{rgb}{0.0, 0.18, 0.39}

\hypersetup{
    plainpages=false,
    hidelinks,
    colorlinks=true,
    urlcolor={coolblack},
    linkcolor={black},
    citecolor={coolblack}
}

\begin{document}
% ----- Pagine di fronespizio, numerate in romano (i,ii,iii,iv...) (obbligatorio)
\frontmatter
\generaFrontespizio
\introduzione{capitoli/abstract}
\generaIndice
\generaIndiceFigure

% ----- Pagine di tesi, numerate in arabo (1,2,3,4,...) (obbligatorio)
\mainmatter
% il comando ``capitolo'' ha come parametri:
% 1) il titolo del capitolo
% 2) il nome del file tex (senza estensione) che contiene il capitolo. Tale nome \`e usato anche come label del capitolo
% \sezione{Nome sezione}{sez1}
\capitolo{Background}{capitoli/background}
\capitolo{Flow}{capitoli/flow}
\capitolo{Implementazione}{capitoli/implementazione}

% Bibliografia con BibTeX (obbligatoria)
% Non si deve specificare lo stile della bibliografia
\bibliography{bibliografia} % inserisce la bibliografia e la prende in questo caso da bibliografia.bib

\end{document}
