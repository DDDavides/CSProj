La \textbf{blockchain} è una struttura dati che consiste in elenchi crescenti di record. denominati blocchi e collegati tra loro in modo sicuro, verificabile e permanente utilizzando la crittografia. Ogni blocco contiene oltre ad un \textit{timestamp} e i dati di transazione, un hash crittografico del blocco precedente e per tale motivo le transazioni sono irreversibili in quanto i dati in un determinato blocco non possono essere modificati retroattivamente senza alterare tutti i blocchi successivi. 
Una caratteristica fondamentale di tale struttura dati è quella della decentralizzazione per cui ogni nodo nel sistema decentrato ha una copia della blockchain: difatti la qualità dei dati è mantenuta grazie a una massiva replicazione del database e non esiste nessuna copia ufficiale centralizzata e nessun utente risulta più credibile di altri.

Esistono principalmente due tipologie fondamentali di blockchain:
\begin{itemize}
    \item \textit{permissionless} o pubblica per cui non si necessita di nessuna protezione verso utenti malintenzionati e di nessun controllo degli accessi;
    \item \textit{permissioned} o con permessi le quali sono caratterizzate da un accesso alla rete ristretto ad alcuni partecipanti autorizzati e da un processo di validazione demandato a un gruppo ristretto di attori.
\end{itemize}

Inoltre parte fondamentale nelle blockchain risultano essere gli \textit{\textbf{smart contracts}}, un programma che viene messo in esecuzione sui nodi validatori e il cui risultato, che in genere corrisponde ad un cambio di stato della blockchain stessa, rappresenta una transazione sulla quale i nodi validatori devono trovare un consenso.

Infine esempi di blockchain di interesse sono quelle di \textbf{Bitcoin}, \textbf{Ethereum}, \textbf{Algorand} e \textbf{Flow} la quale verrà mostrata più nel dettaglio nel capitolo \ref{cap:Flow}coglione